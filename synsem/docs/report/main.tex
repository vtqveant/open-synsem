\documentclass[a4paper,12pt]{article}

%%% Поля и разметка страницы %%%
\usepackage{lscape} % Для включения альбомных страниц
\usepackage{afterpage}
\usepackage{geometry} % Для последующего задания полей

%%% Кодировки и шрифты %%%
\usepackage[T2A]{fontenc}
\usepackage[utf8]{inputenc}
\usepackage[russian,english]{babel}

%%% Математические пакеты %%%
\usepackage{amsthm,amsfonts,amsmath,amssymb,amscd} % Математические дополнения от AMS

%%% Оформление абзацев %%%
\usepackage{indentfirst} % Красная строка

%%% Таблицы %%%
\usepackage{longtable} % Длинные таблицы
\usepackage{multirow,makecell,array} % Улучшенное форматирование таблиц

%%% Общее форматирование
\usepackage[singlelinecheck=off,center]{caption} % Многострочные подписи
\usepackage{soul} % Поддержка переносоустойчивых подчёркиваний и зачёркиваний

%%% Библиография %%%
\usepackage{cite}

%%% Гиперссылки %%%
\usepackage[plainpages=false,pdfpagelabels=false]{hyperref}

%%% Изображения %%%
\usepackage{graphicx} 

\usepackage{graphicx}
\usepackage{booktabs}
\usepackage{tabularx}

%%% Макет страницы %%%
\geometry{a4paper,top=3cm,bottom=3cm,left=3.5cm,right=3cm}

%%% Кодировки и шрифты %%%
% \renewcommand{\rmdefault}{ftm} % Включаем Times New Roman

%%% Выравнивание и переносы %%%
\sloppy
\clubpenalty=10000
\widowpenalty=10000

%%% Библиография %%%
%\makeatletter
%\bibliographystyle{utf8gost705u} % Оформляем библиографию в соответствии с ГОСТ 7.0.5
%\renewcommand{\@biblabel}[1]{#1.} % Заменяем библиографию с квадратных скобок на точку:
%\makeatother

%%% Изображения %%%
\graphicspath{{images/}} % Пути к изображениям

%% CCG parse trees
\usepackage[inference]{semantic}

%% regular trees (e.g. GB style)
\usepackage{qtree}

%% numbered lists for linguistic examples
\usepackage{gb4e}

%% diagrams
\usepackage{xypic}


\begin{document}

\begin{titlepage}

\begin{center}
\vfill
%\framepage

{\Large NLU/RG}\\

\vfill

\ \\

{\large\bf Интерпретация естественно-языковых запросов с помощью гибридной логики и проверки моделей}
\ \\
К. В. Соколов\\

\vfill
\vfill

Санкт-Петербург, 2014
\end{center}

\end{titlepage}


\section{Введение}

Предмет исследования\\
Цель исследования\\
Задачи исследования\\

\section{Постановка задачи и анализ}

\subsection{Постановка задачи}
\begin{itemize}
  \item Умный дом
	\begin{itemize}
  		\item согласованное взаимодействие разнородного оборудования
		\item обратная связь, видимость, прозрачность
		\item персонализированное, адаптивное, проактивное поведение
	\end{itemize}
  \item Мультимодальное управление
	\begin{itemize}
  		\item сосуществование различных способов управления
		\item естественно-языковой интерфейс не должен уметь всё
	\end{itemize}
\end{itemize}


\subsection{Допущения и предпосылки}
Командный режим работы:
\begin{itemize}
	\item Это не диалоговая система и не IVR
	\item Обратная связь
		\begin{itemize}
			\item изменение состояния устройств 
			\item панель управления
			\item носимые устройства 
			\item звуковые сигналы и пр.
		\end{itemize}
	\item Голосовое взаимодействие одностороннее
		\begin{itemize}
			\item уточнений и наводящих вопросов от системы нет
			\item синтеза речи нет
			\item \textit{mixed initiative} нет
		\end{itemize}
\end{itemize}

Контролируемый язык:\\
\begin{itemize}
	\item Две крайности, которых хотелось бы избежать:
		\begin{itemize}
			\item всё что угодно 
			\item список команд для заучивания
		\end{itemize}
	\item Ограниченный домен
		\begin{itemize}
			\item мы работаем с очень узкой предметной областью
			\item мы контролируем набор языковых конструкций и словарь
		\end{itemize}
	\item Цель: говорить как угодно, но ''только про лампочки''.
\end{itemize}

Дискурс:\\
\begin{itemize}
	\item Поддержка контекста взаимодействия в пределах сессии 
	\item В пределах сессии пользователь может
		\begin{itemize}
			\item чего-то не уточнять, предполагать известным
			\item ссылаться на предшествующее состояние системы
			\item ссылаться на свои прошлые слова
		\end{itemize}
\end{itemize}

Голосовое управление:\\
\begin{itemize}
	\item ASR может ошибаться
	\item ASR может возвращать список гипотез
	\item ASR можно настраивать с учетом текущего контекста
\end{itemize}

Ежедневное использование:\\
\begin{itemize}
	\item Пользователей мало, но они взаимодействуют постоянно
	\item Доля успешных взаимодействий должна быть большой
	\item Требуется качественный анализ и быстродействие
\end{itemize}

Технические особенности:\\
\begin{itemize}
	\item Возможность опереться на знания о системе и пользователях
	\item Возможность использовать специализированное оборудование для решения задачи управления
	\item Возможность использовать вычислительно тяжелые подходы
	\item Нет возможности ''ходить в облако''
\end{itemize}

\subsection{Близкие области исследований}
\begin{itemize}
	\item Вычислительная семантика
		\begin{itemize}
			\item теоретико-модельная формальная семантика
			\item синтактико-семантический интерфейс
			\item NLU как компонент систем распознавания речи
		\end{itemize}
	\item Искусственный интеллект
		\begin{itemize}
			\item символьные методы в компьютерной лингвистике
			\item робототехника (анализ сцен, восприятие)
			\item представление знаний и \textit{interactive grounding}
		\end{itemize}
	\item Computer Science
		\begin{itemize}
			\item логический вывод
			\item проверка и построение моделей
			\item верификация программ
		\end{itemize}
	\item Известные решения
		\begin{itemize}
			\item C\&C Tools
			\item OpenCCG
			\item LKB, PATR-II и др.
		\end{itemize}
	\bigskip
	\item Дорожки
		\begin{itemize}
			\item Recognizing Textual Entailment (c 2005 г.)
			\item Supervised Semantic Parsing of Robotic Spatial Commands (SemEval-2014)
		\end{itemize}
\end{itemize}


\subsection{Основные проблемы}
\begin{itemize}
	\item Ошибки распознавания
		\begin{itemize}
			\item \texttt{``включи''} и \texttt{``выключи''}
			\item \texttt{``включи свет и лампу''} и \texttt{``включи свет у лампы''}
		\end{itemize}
		\bigskip
	\item Неоднозначность
		\begin{itemize}
			\item \texttt{``красная и белая лампа''}
				\begin{itemize}
					\item две лампы, одна красная и одна белая
					\item одна красно-белая лампа (``большой и сильный человек'')
				\end{itemize}
			\item \texttt{``красная лампа в прихожей''}
				\begin{itemize}
					\item в прихожей много ламп, одна из них красная
					\item красных ламп много, одна из них в прихожей
				\end{itemize}
			\item \texttt{``включи свет и телевизор на кухне''}
				\begin{itemize}
					\item и свет, и телевизор - на кухне
					\item телевизор на кухне, свет - нет
				\end{itemize}
		\end{itemize}
	\item Учет семантических и прагматических факторов
		\begin{itemize}
			\item \texttt{``включи лампу на кухне и телевизор в комнате''}
				\begin{itemize}
					\item кухня не может стоять в комнате
				\end{itemize}
			\item \texttt{``включи лампу на столе и телевизор в комнате''}
				\begin{itemize}
					\item лампа на столе, стол в комнате
					\item лампа на столе, но стол не в комнате
				\end{itemize}
			\item \texttt{``включи лампу''}
				\begin{itemize}
					\item зависимость от местоположения
					\item зависимость от конфигурации (несколько ламп)
				\end{itemize}
			\item \texttt{``сделай музыку погромче''}
				\begin{itemize}
					\item разное поведение днем и ночью
					\item различные пользовательские предпочтения
				\end{itemize}
		\end{itemize}
\end{itemize}






\section{Описание выбранного подхода}

\subsection{Intro}
Синтаксический анализ средствами мульитмодальных комбинаторных категориальных грамматик (MMCCG) с трансляцией в семантическое представление в формализме семантики зависимостей на основе гибридной логики (HLDS). Экспликация синтаксической и семантической неоднозначности в виде списка гипотез. Построение гибридной модели предметной области на основе файлов конфигурации и фрагментов семантических представлений текстовых описаний. Трансляция семантического представления естественно-языкового запроса в выражение на языке гибридной логики с учетом семантических ролей. Разрешение референции и оценка выполнимости запросов в текущей модели средствами проверки моделей (алгоритм гибридной проверки моделей MCFull). Оценка гипотез с точки зрения структурной сложности деревьев синтаксического разбора, семантических представлений и выполнимости в модели. Передачи гипотез с оценками в модуль принятия решений.


\subsection{Предлагаемый подход}

\begin{itemize}
	\item Синтактико-семантический компонент
		\begin{itemize}
			\item получает список вариантов от ASR
			\item формирует варианты интерпретации
			\item устраняет заведомо некорректные варианты
			\item расширяет пространство гипотез, делая допущения
			\item вычисляет оценки правдоподобности гипотез
			\item формирует ранжированный список гипотез
		\end{itemize}
	\item Модуль принятия решений делает выбор на основе 
		\begin{itemize}
			\item прагматической информации
				\begin{itemize}
					\item местоположения пользователя
					\item состояния оборудования
					\item времени суток
				\end{itemize}
			\item эвристик и правила (интеллектуальное поведение)
			\item модели пользователя (персонализация) 
			\item статистики прошлых запросов (адаптация)
			\item текущих предпочтений (режим работы, энергосбережение)
		\end{itemize}
\end{itemize}		


Суть подхода - генерация гипотез и фильтрация\\

\resizebox{1.0\linewidth}{!}{ % Resize table to fit within \linewidth horizontally
    \begin{tabularx}{1.7\textwidth}{lcX}
    \toprule
    \textbf{Компонент}          & \textbf{Число гипотез}  & \textbf{Параметры для расчета оценки}\\
    \midrule
    Распознавание речи          & 10             & score, confidence\\ 
    Расширение запроса          & 100            & количество вставок, удалений, пропусков, словарных замен,\newline 
    											   расстояние редактирования\\ 
    Парсер                      & 100            & сложность разборов, проективность, использование доп. правил \\ 
    Логическая форма            & 200            & глубина для рекурсивных структур, количество переменных,\newline
    											   число клауз и пр.\\ 
    Оценка модели               & 10             & выполнимость всей формы или отдельных её частей\\
    \bottomrule
    \end{tabularx}
}


\subsection{Цели прототипирования}

\begin{itemize}
	\item Работа с русским языком
	\item Работа с неоднозначностью
	\item Оценка реализуемости и трудоемкости
	\item Оценка возможностей имеющихся open-source решений для работы с deep semantic parsing
	\item Оценка возможности интеграции в существующую инфраструктуру (включая интеграцию с ASR)
	\item Оценка вариативности языковых конструкций, поддающихся реализации
\end{itemize}

НЕ ставились цели:\\
\begin{itemize}
	\item Достичь быстродействия системы или оценить его
	\item Покрытие кейсов по сценариям использования системы конечным пользователем
\end{itemize}

\subsection{Цели демонстрации}
\begin{itemize}
	\item Рассказать о теоретических основах подхода
	\item Рассказать об используемых компонентах
	\item Рассказать об используемых формализмах
	\item Посмотреть на реализацию разборов \\конкретных типов предложений
	\item Дать оценки по исходным задачам прототипа
	\item Рассказать об ограничених текущего прототипа и как их можно решать
	\item Рассказать, что еще не было сделано
	\item Рассказать о дополнительных возможностях, которые может предостаавить выбранный подход
\end{itemize}



\section{Обсуждение}

\subsection{Теорминимум}

Варианты логического описания структр зависимостей:
\begin{itemize}
    \item логика Каспера-Раундса
	\item гибридная логика и HLDS
	\item dependency grammar logic (Kruijff, phd thesis)
\end{itemize}
 
\bigskip

Что-то про модел-чекинг и его обоснование.\\
Обзор моделчекеров (HLMC, HTab, HyLoRes, Spartacus) и используемые в них подходы\\
Вычислительная сложность задачи модел-чекинга для HL.
Портирование HLMC, обоснование необходимости портирования.\\

\bigskip

Выбранный подход
\begin{itemize}
	\item Multimodal Combinatory Categorial Grammar (MMCCG)
	\item Hybrid Logic Dependency Semantics (HLDS)
	\item Hybrid Logic Model Checking (HLMC)
\end{itemize}

Hybrid Logic Dependency Semantics:\\
\begin{itemize}
	\item Композициональный семантический формализм 
	\item Описание семантических структур зависимостей с помощью выражений гибридной логики
	\item Cемантическая композиция реализуется как унификация логических форм (ср. с формализмами на основе $\lambda$-исчисления: конкатенация с последующей редукцией)
	\item Реализован в системе OpenCCG (Baldridge et al., 2007)
\end{itemize}

Технологии\\
\begin{itemize}
	\item OpenCCG (Baldridge et al., 2007)
	\item Грамматика Moloko (DFKI)
	\item HLMC (L. Dragone)
\end{itemize}


Формализмы\\
DotCCG
\begin{itemize}
	\item DSL для создания MMCCG-грамматик
	\item Транслируется в XML-формат OpenCCG
	\item MOLOKO - около 4 kLOC 
	\item Мой прототип - около 0.5 kLOC (из-за морфологии)
\end{itemize}





\section{Заключение}

\subsection{Работа с русским языком}
\begin{itemize}
	\item Простые типы предложений портируются легко
	\item Необходимо аккуратно реализовать онтологию
	\item Трудоемкая реализация морфологии (но ср.: RusForIR)
	\item Программировать не нужно
	\item Работа для лингвиста (или нескольких)
	\item Отличные возможности для повторного использования
	\item Реализация большого фрагмента русского языка в формализме MMCCG 
может стать существенным вкладом в отечественную компьютерную лингвистику
\end{itemize}

\subsection{Работа с неоднозначностью}
\begin{itemize}
	\item Удалось реализовать все предполагавшиеся варианты
	\item ... и обнаружить еще несколько
\end{itemize}

\subsection{Что предстоит сделать}
\begin{itemize}
	\item Поддержка дискурса
	\item Расчет оценок
	\item Портирование HLMC и поддержка типов
	\item Автоматизация создания модели
\end{itemize}

\subsection{Новые возможности}
\begin{itemize}
	\item Генерация правого контекста для распознавания речи (todo)
	\item Валидация модели  (есть)
	\item Извлечение семантических ролей  (есть)
\end{itemize}


\nocite{*}  % list all entries without citing
\bibliographystyle{utf8gost705u} 
\bibliography{bib/bibliography}

\end{document}
